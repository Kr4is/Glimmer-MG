\documentclass[fleqn,titlepage,11pt]{article}

\usepackage{latexsym,delcher}

\PortraitPage
\def\baselinestretch{1.0}
\def\thefootnote{\fnsymbol{footnote}}
\def\thepage{{\footnotesize\arabic{page}}}
\def\today{17~May~2011}

\def\Desc#1{\,\mbox{\emph{#1}}\,}
\def\Glimmer{\textsc{Glimmer}}
\def\Gtwo{\textsc{Glimmer2}}
\def\Gthree{\textsc{Glimmer3}}
\def\Gmg{\textsc{Glimmer-MG}}
\def\Phymm{\textsc{Phymm}}
\def\Scimm{\textsc{Scimm}}
\def\PhyScimm{\textsc{PhyScimm}}
\def\PgBICM{\texttt{build-icm}}
\def\Pg#1{\texttt{#1}}


\begin{document}

\RaggedRight
\sloppy

\title{\Gmg{} Release Notes \\ Version~0.1}
\author{David R. Kelley, Art L. Delcher}
\titlepagenote{Copyright \copyright\ 2011 University of Maryland Center for Bioinformatics \& Computational Biology}

\maketitle

%%%%%%%%%%%%%%%%%%%%%%%%%%%%%%%%%%%%%%%%%%%%%%%%%%%%%%%%%%%%
\section{Introduction}

This document describes Version~0.1 of the \Gmg{} gene-finding
software for metagenomic sequences.  Users discovering problems or
errors are encouraged to report them to
\,\verb`dakelley@umiacs.umd.edu`\,.

\Glimmer{} is a collection of programs for identifying genes in
microbial DNA sequences.  The system works by creating a
variable-length Markov model from a training set of genes and then
using that model to attempt to identify all genes in a given DNA
sequence.  Version~1 of \Glimmer{} was described in~\cite{glimmer1},
Version~2 was described in~\cite{glimmer2}. and Version~3 was
described in~\cite{glimmer3}.

\Gmg{} is released as OSI Certified Open Source Software under the
Artistic License.  The license is contained in the file, \Pg{LICENSE},
in the distribution.

%%%%%%%%%%%%%%%%%%%%%%%%%%%%%%%%%%%%%%%%%%%%%%%%%%%%%%%%%%%%
\section{Installation}

\Gmg{} software was written for the Linux software environment.  The
following instructions assume a Linux system.  They also work under
Mac OSX.

To install \Gmg{}, download the compressed tarfile
\,\verb`glimmer-mg-1.0.1.tar.gz`\,
from the website.  Then uncompress the file by typing
\BSV\begin{verbatim}
  tar xzf glimmer-mg.1.0.1.tar.gz
\end{verbatim}\ESV
A directory named \,\verb`glimmer-mg`\, should result.
In that directory, is a subdirectory named \,\verb`src`\,.
Within the \,\verb`src`\, subdirectory type
\BSV\begin{verbatim}
  make
\end{verbatim}\ESV
(or alternately \Pg{gmake}).
This will compile the \Gmg{} programs
and put the executable files in the directory
\,\verb`glimmer-mg/bin`\,.  These files can be copied or moved to
whatever directory is convenient to the user.

%%%%%%%%%%%%%%%%%%%%%%%%%%%%%%%%%%%%%%%%%%%%%%%%%%%%%%%%%%%%
\section{Running \Gmg{}}
\Gmg{} can be run in a few different modes, depending on the
characteristics of the sequences provided, e.g., whether they
represent accurate contigs or reads from the Illumina or 454
technologies. Then within each mode, \Gmg{} can be run with varying
amounts of classification/clustering preprocessing.

\subsection{Preprocessing options}
The \Gmg{} pipeline can be run a number of different ways, including a
Python script to run the full pipeline and options to classify/cluster
the reads separately and predict genes using those results.

% Running everything
\subsubsection{Running from scratch}
The Python script \,\Pg{glimmer-mg.py}\, runs the full \Gmg{}
pipeline.
\BSV
\begin{verbatim}
glimmer-mg.py seqs.fasta
\end{verbatim}
\ESV
This script will classify the sequences with \Phymm{}, make initial
predictions for all sequences, cluster the sequences with \Scimm{},
and make final predictions within each cluster.

% Running everything, no re-predictions
\subsubsection{Running from scratch with no retraining}
By passing the option \,\Pg{--iter 0}\, to \,\Pg{glimmer-mg.py}\,, the
clustering and re-training steps will be skipped. That is, the
sequences will be classified with \Phymm{} and predictions made.
\BSV
\begin{verbatim}
glimmer-mg.py --iter 0 seqs.fasta
\end{verbatim}
\ESV

% Classifying first, then running
\subsubsection{Classification separate}
Some users may prefer to run the computationally intensive
classification of sequences separately (e.g. on a computer cluster).
\,\Pg{glimmer-mg.py}\, can be made to expect this by using the
\,\Pg{--raw}\, option, which specifies that the raw \Phymm{} output file
already exists in the current working directory.
\BSV
\begin{verbatim}
glimmer-mg.py --raw seqs.fasta
\end{verbatim}
\ESV

% Classifying and clustering first, then running
\subsubsection{Clustering separate}
In a similar vein, clustering can be performed separately before
making gene predictions by specifying the \,\Pg{--clust}\, option,
which specifies that \Scimm{} or \PhyScimm{} output is in the current
working directory.
\BSV
\begin{verbatim}
glimmer-mg.py --clust seqs.fasta
\end{verbatim}
\ESV

\subsection{Sequence modes}
Depending on the error characteristics of the input sequences, \Gmg{}
has three modes that best handle certain types of data. In each case,
the options described can be passed to the C++ binary
\,\Pg{glimmer-mg}\, or the Python script to manage the entire pipeline
\,\Pg{glimmer-mg.py}\,.

\subsubsection{Accurate contigs}
If you believe that the sequences on which you would like to find
genes are very accurate, than run \Gmg{} in the default mode with no
additional options.

\subsubsection{454 indels}
If the input sequences are reads or contigs from the 454 or Ion
Torrent technologies and indel errors are expected to exist, \Gmg{}
can be made to predict such errors and account for them in its gene
predictions by using \emph{indel} mode.
\BSV
\begin{verbatim}
glimmer-mg.py --indels 454.fasta
\end{verbatim}
\ESV

\subsubsection{Illumina substitution errors}
If the input sequences are reads or contigs from the Illumina
technology and substitution errors are expected to exist, \Gmg{}
can be made to predict such errors and account for them in its gene
predictions by using \emph{substitution} mode.
\BSV
\begin{verbatim}
glimmer-mg.py --sub illumina.fasta
\end{verbatim}
\ESV

%%%%%%%%%%%%%%%%%%%%%%%%%%%%%%%%%%%%%%%%%%%%%%%%%%%%%%%%%%%%
\section{Sample Run Directory}

A directory containing a sample run of \Gthree{} is provided.
This directory, named \,\Pg{sample-run}\, contains the genome sequence
for \emph{Treponema pallidum} (file \,\Pg{tpall.fna}\,)
and a list of annotated genes for it (file \,\Pg{tpall.nh}\,),
both downloaded from GenBank.
The files whose names begin \,\Pg{from-scratch}\, are the result of
running the script
\BSV
\begin{verbatim}
  g3-from-scratch.csh tpall.fna from-scratch
\end{verbatim}
\ESV
The files whose names begin \,\Pg{from-training}\, are the result of
running the script
\BSV
\begin{verbatim}
  g3-from-training.csh tpall.fna tpall.nh from-training
\end{verbatim}
\ESV
The files whose names begin \,\Pg{iterated}\, are the result of
running the script
\BSV
\begin{verbatim}
  g3-iterated.csh tpall.fna iterated
\end{verbatim}
\ESV
Users will need to modify the path directories at the top of these
scripts to be able to run them (see Section~\ref{script:sec} above).

%%%%%%%%%%%%%%%%%%%%%%%%%%%%%%%%%%%%%%%%%%%%%%%%%%%%%%%%%%%%
\section{Notes on \Gmg{} programs}
\subsection{\Pg{glimmer-mg} Program}

This is the main program that makes gene preditions.

\subsubsection{\Pg{glimmer-mg} Parameters \& Options}
The invocation for \,\Pg{glimmer3}\, is:
\bq
  \Pg{glimmer3}\, [\Desc{options}] \Desc{sequence} \Desc{icm} \Desc{tag}
\eq
where \Desc{sequence} is the name of the file containing the DNA
sequence(s) to be analyzed and \Desc{icm} is the name of the file
containing the ICM model produced by \,\verb`build-icm`\,.  \Desc{tag}
is a prefix used to name the two output files:  \Desc{tag}\verb`.detail`
and \Desc{tag}\verb`.predict`.

\Desc{options} can be the following:
\bl{}\RaggedRight
\exdent
  \verb`-A` \Desc{codon-list} \enskip or \enskip \verb`--start_codons` \Desc{codon-list}

  Specify start codons as a comma-separated list.
  Sample format:  \,\verb`-A atg,gtg`\,.
  The default start codons are \Pg{atg}, \Pg{gtg} and \Pg{ttg}.
  Use the \Pg{-P} option to specify the relative proportions of use.
  If \Pg{-P} is not used, then the proportions will be equal.

\exdent
  \verb`-b` \Desc{filename} \enskip or \enskip \verb`--rbs_pwm` \Desc{filename}

  Read a position weight matrix (PWM) from \Desc{filename} to identify
  the ribosome binding site to help choose start sites.  The format of
  this file is indicated by the following example:
\BSV
\begin{verbatim}
6
a     212     309      43      36     452     138
c      55      58       0      19      48      26
g     247     141     501     523       5     365
t      64      70      34       0      73      49
\end{verbatim}
\ESV
  The first line is the number of positions in the pattern, \ie,
  the number of columns in the matrix (not counting
  the first column of labels).  The column values are the relative
  frequencies of nucleotides at each position.

\exdent
  \verb`-C` \Desc{p} \enskip or \enskip \verb`--gc_percent` \Desc{p}

  Use \Desc{p} as the GC percentage of the independent model, \ie,
  the model of intergenic sequence.
  Note:  \Desc{p} should be a percentage, \eg, \verb`-C 45.2`

  If this option is not specified, the GC percentage will be
  counted from the input file.

\exdent
  \verb`-E` \Desc{filename} \enskip or \enskip \verb`--entropy` \Desc{filename}

  Read entropy profiles from \Desc{filename}.  The format is one header
  line, then 20 lines of 3 columns each, which is the format produced
  by the program \Pg{entropy-profile} with the \Pg{-b} option.
  The columns are amino acid,
  positive entropy, and negative entropy, respectively.  Rows must be in
  alphabetical order by amino acid code letter.  This currently does
  not affect \Gthree{} predictions, but is used in
  the \Pg{long-orfs} program.  If the option is specified, the
  entropy-distance ratio for each potential gene is printed as the last column
  of the \Pg{.detail} file.  If \Desc{filename} is ``\Pg{\#}'', then
  a set of default entropy profiles, constructed from a wide range of
  species, is used.

\exdent
 \verb` -f` \enskip or \enskip \verb`--first_codon`

  Use the first possible codon in an orf as the start codon
  for initial scoring purposes.  Otherwise, the highest-scoring
  codon will be used.  This only affects the start positions in
  the \,\verb`.detail`\, file.  The final start predictions in
  the \,\verb`.predict`\, file are always based on the scoring
  functions.

\exdent
  \verb`-g` \Desc{n} \enskip or \enskip \verb`--gene_len` \Desc{n}

  Set the minimum gene length to \Desc{n} nucleotides.  This does not include
  the bases in the stop codon.

\exdent
  \verb`-h` \enskip or \enskip \verb`--help`

  Print the usage message.

\exdent
  \verb`-i` \Desc{filename} \enskip or \enskip \verb`--ignore` \Desc{filename}

  \Desc{filename} specifies regions of bases that are off 
  limits, so that no bases within that area will be examined.
  The format for entries in this file is one line per region,
  with the start and end positions of the region specified
  as the first two fields on the line.  The rest of the line
  is regarded as comments.  Additionally, any line beginning
  with a \,\verb`#`\, is regarded as a comment.  \Eg, the
  following file:
\BSV
\begin{verbatim}
   1001     1600   Comment here
# The region can be specified high-low as well as low-high
   5600     5001
\end{verbatim}
\ESV
  would ignore bases $1001 \ldots 1600$ and $5001 \ldots 5600$
  in the input sequence.  This option should not be used with
  multi-sequence input files.

\exdent
  \verb`-l` \enskip or \enskip \verb`--linear`

  Assume a linear rather than circular genome, \ie, there will
  be no genes that ``wraparound'' between the beginning and end
  of the sequence.

\exdent
  \verb`-L` \Desc{filename} \enskip or \enskip \verb`--orf_coords` \Desc{filename}

  \Desc{filename} specifies a list of orfs that should
  be scored separately, with no attempt to resolve overlaps or
  determine start codons.  The format of the
  list is one orf per line, with entries separated by white space.
  The first entry is an identifier for the orf.  It can be an
  arbitrary string without spaces.  The next two entries are
  the start and end positions of the orf, respectively, (coordinates counting
  from 1), including the stop codon.  The fourth entry is the
  reading frame.  This is used only to determine the direction of
  the orf in cases of circular genomes where the orf might ``wrap
  around'' the end of the input sequence.  If positive the
  orf is presumed to be on the positive DNA strand; otherwise,
  on the negative strand.  Any further entries on the line are ignored.

  The output with this option goes both to the \Pg{.predict} file
  and to the \Pg{.detail} file.

\exdent
  \verb`-M` \enskip or \enskip \verb`--separate_genes`

  \Desc{sequence-file} is a multifasta file of separate genes to
  be scored separately, with no overlap rules.  Each sequence
  is assumed to be in $5'$ to $3'$ order and to include the stop
  codon.

\exdent
  \verb`-o` \Desc{n} \enskip or \enskip \verb`--max_olap` \Desc{n}

  Set the maximum overlap length to \Desc{n}.  Overlaps of this
  many or fewer bases are allowed between genes.  The new
  dynamic programming algorithm should \underline{\emph{never}}
  output genes that overlap by more than this many bases.

\exdent
  \verb`-P` \Desc{number-list} \enskip or \enskip \verb`--start_probs` \Desc{number-list}

  Specify the probability of different start codons (same number and order
  as in \Pg{-A} option).  If no \Pg{-A} option is given, then there should be 3
  values:  for \Pg{atg}, \Pg{gtg} and \Pg{ttg},
  in that order.  Sample format:  \verb`-P 0.6,0.35,0.05`.
  If \Pg{-A} is specified without \Pg{-P}, then each start codon is equally likely
  (which is very unusual).

\exdent
  \verb`-q` \Desc{n} \enskip or \enskip \verb`--ignore_score_len` \Desc{n}

  Consider any gene \Desc{n} or more bases long as a potential
  gene, regardless of its in-frame score.
  Without this option, this value is calculated automatically to
  be the length such that the expected number of orfs this long
  or longer in a random sequence of a million bases is one.

\exdent
  \verb`-r` \enskip or \enskip \verb`--no_indep`

  Don't use the independent probability score column at all.  Using
  this option will produce more short gene predictions.

\exdent
  \verb`-t` \Desc{n} \enskip or \enskip \verb`--threshold` \Desc{n}

  Set the threshold score for consideration as a gene to \Desc{n}.
  If the in-frame
  score $\ge \Desc{n}$, then the region is given a number and considered
  a potential gene.  Note this is the integer score in the column labelled
  ``InFrm'' in the \,\verb`.detail`\, file, not the decimal score in
  the column labelled ``Raw''.

\exdent
  \verb`-X` \enskip or \enskip \verb`--extend`

  Also score orfs that extend off the end of the sequence(s).  This
  option presumes that the sequence(s) is linear and not circular.
  Reported positions off the end of the sequence are the nearest
  positions in the correct reading frame.  Note that this ignores
  any partial codons at the ends of a sequence.  Suppose, for example,
  that a sequence is 998bp long and an orf in reading frame +1
  starts at position 601 and extends off the end of the sequence.
  Then the end of that gene/orf will be reported at position 999,
  as if the stop codon were in positions 997\ldots999.  This is true
  even if the last two characters of the sequence are, say, \,\verb`cc`\,
  and cannot possibly be part of a stop codon.

  Any scores associated with orfs that extend past the end of a
  sequence are computed using only complete codons contained in
  the sequence.
  
\exdent
  \verb`-z` \Desc{n} \enskip or \enskip \verb`--trans_table` \Desc{n}

  Use Genbank translation table number \Desc{n} to specify stop codons.

\exdent
  \verb`-Z` \Desc{codon-list} \enskip or \enskip \verb`--stop_codons` \Desc{codon-list}

  Specify stop codons as a comma-separated list.
  Sample format:  \verb`-Z tag,tga,taa`.
  The default stop codons are \Pg{tag}, \Pg{tga} and \Pg{taa}.
\el


\subsubsection{\Pg{glimmer3} Output Formats}

\smallskip
\noindent\textbf{\Pg{.detail} File}
\smallskip

The \Pg{.detail} file begins with the command that invoked the program and
a list of the parameters
used by the program.  Here is a sample:
\BSV
\begin{verbatim}
Command:  /fs/szgenefinding/Glimmer3/bin/glimmer3 -o 50 -g 110 -t 30 -b iterated.motif -P 
0.603,0.338,0.059 tpall.fna iterated.icm iterated

Sequence file = tpall.fna
Number of sequences = 1
ICM model file = iterated.icm
Excluded regions file = none
List of orfs file = none
Input is NOT separate orfs
Independent (non-coding) scores are used
Circular genome = true
Truncated orfs = false
Minimum gene length = 110 bp
Maximum overlap bases = 50
Threshold score = 30
Use first start codon = false
Start codons = atg,gtg,ttg
Start probs = 0.603,0.338,0.059
Stop codons = taa,tag,tga
GC percentage = 52.8%
Ignore score on orfs longer than 799
\end{verbatim}
\ESV

Following that, for each sequence in the input file the
fasta-header line is echoed and followed by a list of orfs
that were long enough for \Pg{glimmer3} to score.  Here is
a sample of the beginning of such a section:
\BSV
\begin{verbatim}
>gi|15638995|ref|NC_000919.1| Treponema pallidum subsp. pallidum str. Nichols, complete ge
nome
Sequence length = 1138011

           ----- Start -----           --- Length ----  ------------- Scores -------------
 ID  Frame   of Orf  of Gene     Stop   of Orf of Gene      Raw InFrm F1 F2 F3 R1 R2 R3 NC
        +2       17       20      139      120     117    -4.94     0 99  0  -  0  -  -  0
        +2      140      242      361      219     117     0.99     0 87  0  - 12  -  -  0
        -1      435      417      148      285     267     5.48     2 97  -  -  2  -  -  0
        +2      668      668      790      120     120     2.89     0 99  0  -  -  -  -  0
        -3      899      839      717      180     120    -0.86     1 95  -  -  -  -  1  3
        -1      936      933      808      126     123     0.38    13 78  -  - 13  -  -  8
        -3     1124     1109      918      204     189    -1.32     0 99  -  -  -  -  0  0
0001    +1        4        4     1398     1392    1392     6.61    99 99  -  -  -  -  -  0
        -2     1750     1720     1457      291     261    -0.92     8  -  -  -  -  8  - 91
        -2     1957     1945     1751      204     192    -1.47     1  -  - 70  -  1  - 27
        -3     2078     2063     1908      168     153    -1.88     4  -  - 20  -  -  4 75
        -2     2308     2293     2174      132     117    -0.38     5  -  - 85  -  5  -  9
0002    +3     1542     1641     2756     1212    1113     3.20    99  -  - 99  -  -  -  0
        -3     2807     2774     2616      189     156    -2.08     3  0  -  -  -  -  3 96
\end{verbatim}
\ESV
Below is a description of the columns.  All positions are counted from the beginning of
the sequence with the first base being position~$1$.
\bl{\settowidth{\labelwidth}{Last Column}\leftmargin=\labelwidth \addtolength{\leftmargin}{1em}\labelsep=1em}\RaggedRight
\item[\Pg{ID}]
  An identification number for a potential gene.  Only orfs whose in-frame (\Pg{InFrm})
  score is above the threshold score (set by the \Pg{-t} option) or are longer
  than the ignore-score length have an entry
  in this column.

\item[\Pg{Frame}]
  The reading frame of the orf---positive for forward strand, negative for reverse strand.
  It is determined by the position of the leftmost base of the stop codon:
  \bn
  \exdent
    frame $+1$ if the stop begins in position $1,4,7,\ldots$;
  \exdent
    frame $+2$ if the stop begins in position $2,5,8,\ldots$; 
  \exdent
    frame $+3$ if the stop begins in position $3,5,9,\ldots$; 
  \exdent
    frame $-1$ if the stop begins in position $3,5,9,\ldots$ (so the leftmost base
    is position $1,4,7,\ldots$);
  \exdent
    frame $-2$ if the stop begins in position $4,7,10,\ldots$ (left base position
    $2,5,8,\ldots$);
  \exdent
    frame $-3$ if the stop begins in position $5,8,11,\ldots$ (left base position
    $3,6,9\ldots$).
  \en
  Note that if the genome length is not a multiple of $3$, for genes that wrap
  around the end of the sequence the same rules applied
  to the start codon position will not yield the same reading frame.

\item[\Pg{Start}]
  The positions of the first base of the orf and the first base of the start codon of the
  gene.  Note that the gene start may be different for the same orf in the \Pg{.predict} file.

\item[\Pg{Stop}]
  Position of the last base of the stop codon.

\item[\Pg{Length}]
  Number of bases in the orf and in the gene.  It does \underline{\emph{NOT}}
  include the bases of the stop codon.

\item[\Pg{Raw} Score]
  This is 100 times the per-base log-odds ratio of the in-frame coding ICM score to the
  independent (\ie, non-coding) model score.  It gives a rough quantification to how
  well an orf scores that can be compared between any two orfs.
  
\item[\Pg{InFrm} Score]
  The normalized (to the range $ 0\ldots 99$) score of the gene in its
  reading frame.  This is just the appropriate-frame value among the next
  six scores.

\item[Frame Scores]
  The normalized (to the range $ 0\ldots 99$) score of the gene in each reading frame.
  A ``\Pg{-}'' indicates the presence of a stop codon in that reading frame.
  The normalization compares only scores without stop codons and the independent
  (non-coding) \Pg{NC} score.  If the orf is sufficiently long, \ie, longer than
  the value stated in ``\Pg{Ignore score on orfs longer than}\ldots'',
  the score is not used.

\item[\Pg{NC} Score]
  The normalized independent (\ie, non-coding or intergenic) model score.  This model
  is adjusted for the fact that the orf, by definition, has no in-frame stop codons.

\item[\Pg{EDR} Score]
  An additional column of scores is produced if the \Pg{-E}~option
  is specified.
  This is the entropy-distance ratio, \ie, the ratio of the distance
  of the amino-acid distribution from a positive model to the distance
  from a negative model.  Scores below $1.0$ are more likely to be genes;
  scores above $1.0$ less likely to be genes.
  It is not currently used in the scoring process.
\el

\smallskip
\noindent\textbf{\Pg{.predict} File}
\smallskip

This file has the final gene predictions.  It's format is the fasta-header
line of the sequence followed by one line per gene.  Here is a sample of the
beginning of such a file:
\BSV
\begin{verbatim}
>gms:3447|cmr:632 chromosome 1 {Mycobacterium smegmatis MC2}
orf00001      499     1692  +1    13.14
orf00004     1721     2614  +2    14.20
orf00006     2624     3778  +2    10.35
orf00009     3775     4359  +1     9.34
\end{verbatim}
\ESV
The columns are:
\bl{\settowidth{\labelwidth}{Column 1}\leftmargin=\labelwidth \addtolength{\leftmargin}{1em}\labelsep=1em}\RaggedRight
\item[Column 1]
  The identifier of the predicted gene.  The numeric portion matches the
  number in the \Pg{ID} column of the \Pg{.detail} file.

\item[Column 2]
  The start position of the gene.

\item[Column 3]
  The end position of the gene.  This is the last base of the stop codon, \ie,
  it includes the stop codon.

\item[Column 4]
  The reading frame.

\item[Column 5]
  The per-base ``raw'' score of the gene.  This is slightly different from the
  value in the \Pg{.detail} file, because it includes adjustments for the
  PWM and start-codon frequency.
\el


%%%%%%%%%%%%%%%%%%%%%%%%%%%%%%%%%%%%%%%%%%%%%%%%%%%%%%%%%%%%
\section{Notes on \Glimmer{} programs}

\subsection{\Pg{build-icm} Program}

This program constructs an interpolated context model (ICM)
from an input set of sequences.

\subsubsection{\Pg{build-icm} Parameters \& Options}
The format for invoking \,\Pg{build-icm}\, is:
\bq
  \Pg{build-icm}\, [\Desc{options}] \Desc{output-file} \,\Pg{<}\,\Desc{input-file}
\eq
Sequences are reads from standard input, the ICM is
built and written to \Desc{output-file}.  If \Desc{output-file}
is ``-'', then the output will be sent to standard output.
Since input comes from standard input, one also can ``pipe'' the input
into this program, \eg,
\BSV
\begin{verbatim}
  cat abc.in | build-icm xyz.icm
\end{verbatim}
\ESV
or even type in the input directly.

Possible \Desc{options} are:
\bl{}\RaggedRight
\exdent
  \verb`-d` \Desc{num} \enskip or \enskip \verb`--depth` \Desc{num}

  Set the depth of the ICM to \Desc{num}.  The depth is the
  maximum number of positions in the context window that
  will be used to determine the probability of the predicted
  position.  The default value is 7.

\exdent
  \verb`-F` \enskip or \enskip \verb`--no_stops`

  Do not use any input strings with in-frame stop codons.
  Stop codons are determined by either the \Pg{-z} or \Pg{-Z}
  option.

\exdent
  \verb`-h` \enskip or \enskip \verb`--help`

  Print the usage message.

\exdent
  \verb`-p` \Desc{num} \enskip or \enskip \verb`--period` \Desc{num}

  Set the period of the ICM to \Desc{num}.  The period is the
  number of different submodels for different positions in the
  text in a cyclic pattern.  \Eg, if the period is 3, the first
  submodel will determine positions $1, 4, 7, \dots$; the second
  submodel will determine positions $2, 5, 8, \dots$; and the third
  submodel will determine positions $3, 6, 9, \dots$.  For a
  non-periodic model, use a value of 1.  The default value
  is 3.

\exdent
  \verb`-r` \enskip or \enskip \verb`--reverse`

  Use the reverse of the input strings to build the ICM.  Note that
  this is merely the reverse and \emph{\underline{NOT}} the
  reverse-complement.  In other words, the model is built in
  the backwards direction.

\exdent
  \verb`-t` \enskip or \enskip \verb`--text`

  Output the model in a text format.  This is for
  informational/debugging purposes only---the \Pg{glimmer3}
  program cannot read models in this form.

  The format of the output is a header line containing the
  parameters of the model, followed by individual
  probability lines.  The entries on each probability line
  are:
  \bq
    \begin{tabular}{cl}
      Column & \quad Description \\
      1 & ID number \\
      2 & Context pattern \\
      3 & Mutual information \\
      4 & Probability of A \\
      5 & Probability of C \\
      6 & Probability of G \\
      7 & Probability of T
    \end{tabular}
  \eq
  The context pattern is divided into codons by the vertical lines (this
  option assumes the default 3-periodic model).
  The ``?'' represents the position being predicted.  Letters represent
  specific values in their respective positions in the context window.
  The asterisk indicates the position that has maximum mutual information
  with the predicted position.

\exdent
  \verb`-v` \Desc{num} \enskip or \enskip \verb`--verbose` \Desc{num}

  Set the verbose level to \Desc{num}.  This controls extra debugging
  output---the higher the value the more output.

\exdent
  \verb`-w` \Desc{num} \enskip or \enskip \verb`--width` \Desc{num}

  Set the width of the ICM to \Desc{num}.  The width includes
  the predicted position.  The default value is 12.

\exdent
  \verb`-z` \Desc{n} \enskip or \enskip \verb`--trans_table` \Desc{n}

  Use Genbank translation table number \Desc{n} to specify stop codons.

\exdent
  \verb`-Z` \Desc{codon-list} \enskip or \enskip \verb`--stop_codons` \Desc{codon-list}

  Specify stop codons as a comma-separated list.
  Sample format:  \,\verb`-Z tag,tga,taa`\,.
  The default stop codons are \Pg{tag}, \Pg{tga} and \Pg{taa}.
\el

\subsection{\Pg{glimmer3} Program}

This is the main program that makes gene preditions.

\subsubsection{\Pg{glimmer3} Parameters \& Options}
The invocation for \,\Pg{glimmer3}\, is:
\bq
  \Pg{glimmer3}\, [\Desc{options}] \Desc{sequence} \Desc{icm} \Desc{tag}
\eq
where \Desc{sequence} is the name of the file containing the DNA
sequence(s) to be analyzed and \Desc{icm} is the name of the file
containing the ICM model produced by \,\verb`build-icm`\,.  \Desc{tag}
is a prefix used to name the two output files:  \Desc{tag}\verb`.detail`
and \Desc{tag}\verb`.predict`.

\Desc{options} can be the following:
\bl{}\RaggedRight
\exdent
  \verb`-A` \Desc{codon-list} \enskip or \enskip \verb`--start_codons` \Desc{codon-list}

  Specify start codons as a comma-separated list.
  Sample format:  \,\verb`-A atg,gtg`\,.
  The default start codons are \Pg{atg}, \Pg{gtg} and \Pg{ttg}.
  Use the \Pg{-P} option to specify the relative proportions of use.
  If \Pg{-P} is not used, then the proportions will be equal.

\exdent
  \verb`-b` \Desc{filename} \enskip or \enskip \verb`--rbs_pwm` \Desc{filename}

  Read a position weight matrix (PWM) from \Desc{filename} to identify
  the ribosome binding site to help choose start sites.  The format of
  this file is indicated by the following example:
\BSV
\begin{verbatim}
6
a     212     309      43      36     452     138
c      55      58       0      19      48      26
g     247     141     501     523       5     365
t      64      70      34       0      73      49
\end{verbatim}
\ESV
  The first line is the number of positions in the pattern, \ie,
  the number of columns in the matrix (not counting
  the first column of labels).  The column values are the relative
  frequencies of nucleotides at each position.

\exdent
  \verb`-C` \Desc{p} \enskip or \enskip \verb`--gc_percent` \Desc{p}

  Use \Desc{p} as the GC percentage of the independent model, \ie,
  the model of intergenic sequence.
  Note:  \Desc{p} should be a percentage, \eg, \verb`-C 45.2`

  If this option is not specified, the GC percentage will be
  counted from the input file.

\exdent
  \verb`-E` \Desc{filename} \enskip or \enskip \verb`--entropy` \Desc{filename}

  Read entropy profiles from \Desc{filename}.  The format is one header
  line, then 20 lines of 3 columns each, which is the format produced
  by the program \Pg{entropy-profile} with the \Pg{-b} option.
  The columns are amino acid,
  positive entropy, and negative entropy, respectively.  Rows must be in
  alphabetical order by amino acid code letter.  This currently does
  not affect \Gthree{} predictions, but is used in
  the \Pg{long-orfs} program.  If the option is specified, the
  entropy-distance ratio for each potential gene is printed as the last column
  of the \Pg{.detail} file.  If \Desc{filename} is ``\Pg{\#}'', then
  a set of default entropy profiles, constructed from a wide range of
  species, is used.

\exdent
 \verb` -f` \enskip or \enskip \verb`--first_codon`

  Use the first possible codon in an orf as the start codon
  for initial scoring purposes.  Otherwise, the highest-scoring
  codon will be used.  This only affects the start positions in
  the \,\verb`.detail`\, file.  The final start predictions in
  the \,\verb`.predict`\, file are always based on the scoring
  functions.

\exdent
  \verb`-g` \Desc{n} \enskip or \enskip \verb`--gene_len` \Desc{n}

  Set the minimum gene length to \Desc{n} nucleotides.  This does not include
  the bases in the stop codon.

\exdent
  \verb`-h` \enskip or \enskip \verb`--help`

  Print the usage message.

\exdent
  \verb`-i` \Desc{filename} \enskip or \enskip \verb`--ignore` \Desc{filename}

  \Desc{filename} specifies regions of bases that are off 
  limits, so that no bases within that area will be examined.
  The format for entries in this file is one line per region,
  with the start and end positions of the region specified
  as the first two fields on the line.  The rest of the line
  is regarded as comments.  Additionally, any line beginning
  with a \,\verb`#`\, is regarded as a comment.  \Eg, the
  following file:
\BSV
\begin{verbatim}
   1001     1600   Comment here
# The region can be specified high-low as well as low-high
   5600     5001
\end{verbatim}
\ESV
  would ignore bases $1001 \ldots 1600$ and $5001 \ldots 5600$
  in the input sequence.  This option should not be used with
  multi-sequence input files.

\exdent
  \verb`-l` \enskip or \enskip \verb`--linear`

  Assume a linear rather than circular genome, \ie, there will
  be no genes that ``wraparound'' between the beginning and end
  of the sequence.

\exdent
  \verb`-L` \Desc{filename} \enskip or \enskip \verb`--orf_coords` \Desc{filename}

  \Desc{filename} specifies a list of orfs that should
  be scored separately, with no attempt to resolve overlaps or
  determine start codons.  The format of the
  list is one orf per line, with entries separated by white space.
  The first entry is an identifier for the orf.  It can be an
  arbitrary string without spaces.  The next two entries are
  the start and end positions of the orf, respectively, (coordinates counting
  from 1), including the stop codon.  The fourth entry is the
  reading frame.  This is used only to determine the direction of
  the orf in cases of circular genomes where the orf might ``wrap
  around'' the end of the input sequence.  If positive the
  orf is presumed to be on the positive DNA strand; otherwise,
  on the negative strand.  Any further entries on the line are ignored.

  The output with this option goes both to the \Pg{.predict} file
  and to the \Pg{.detail} file.

\exdent
  \verb`-M` \enskip or \enskip \verb`--separate_genes`

  \Desc{sequence-file} is a multifasta file of separate genes to
  be scored separately, with no overlap rules.  Each sequence
  is assumed to be in $5'$ to $3'$ order and to include the stop
  codon.

\exdent
  \verb`-o` \Desc{n} \enskip or \enskip \verb`--max_olap` \Desc{n}

  Set the maximum overlap length to \Desc{n}.  Overlaps of this
  many or fewer bases are allowed between genes.  The new
  dynamic programming algorithm should \underline{\emph{never}}
  output genes that overlap by more than this many bases.

\exdent
  \verb`-P` \Desc{number-list} \enskip or \enskip \verb`--start_probs` \Desc{number-list}

  Specify the probability of different start codons (same number and order
  as in \Pg{-A} option).  If no \Pg{-A} option is given, then there should be 3
  values:  for \Pg{atg}, \Pg{gtg} and \Pg{ttg},
  in that order.  Sample format:  \verb`-P 0.6,0.35,0.05`.
  If \Pg{-A} is specified without \Pg{-P}, then each start codon is equally likely
  (which is very unusual).

\exdent
  \verb`-q` \Desc{n} \enskip or \enskip \verb`--ignore_score_len` \Desc{n}

  Consider any gene \Desc{n} or more bases long as a potential
  gene, regardless of its in-frame score.
  Without this option, this value is calculated automatically to
  be the length such that the expected number of orfs this long
  or longer in a random sequence of a million bases is one.

\exdent
  \verb`-r` \enskip or \enskip \verb`--no_indep`

  Don't use the independent probability score column at all.  Using
  this option will produce more short gene predictions.

\exdent
  \verb`-t` \Desc{n} \enskip or \enskip \verb`--threshold` \Desc{n}

  Set the threshold score for consideration as a gene to \Desc{n}.
  If the in-frame
  score $\ge \Desc{n}$, then the region is given a number and considered
  a potential gene.  Note this is the integer score in the column labelled
  ``InFrm'' in the \,\verb`.detail`\, file, not the decimal score in
  the column labelled ``Raw''.

\exdent
  \verb`-X` \enskip or \enskip \verb`--extend`

  Also score orfs that extend off the end of the sequence(s).  This
  option presumes that the sequence(s) is linear and not circular.
  Reported positions off the end of the sequence are the nearest
  positions in the correct reading frame.  Note that this ignores
  any partial codons at the ends of a sequence.  Suppose, for example,
  that a sequence is 998bp long and an orf in reading frame +1
  starts at position 601 and extends off the end of the sequence.
  Then the end of that gene/orf will be reported at position 999,
  as if the stop codon were in positions 997\ldots999.  This is true
  even if the last two characters of the sequence are, say, \,\verb`cc`\,
  and cannot possibly be part of a stop codon.

  Any scores associated with orfs that extend past the end of a
  sequence are computed using only complete codons contained in
  the sequence.
  
\exdent
  \verb`-z` \Desc{n} \enskip or \enskip \verb`--trans_table` \Desc{n}

  Use Genbank translation table number \Desc{n} to specify stop codons.

\exdent
  \verb`-Z` \Desc{codon-list} \enskip or \enskip \verb`--stop_codons` \Desc{codon-list}

  Specify stop codons as a comma-separated list.
  Sample format:  \verb`-Z tag,tga,taa`.
  The default stop codons are \Pg{tag}, \Pg{tga} and \Pg{taa}.
\el


\subsubsection{\Pg{glimmer3} Output Formats}

\smallskip
\noindent\textbf{\Pg{.detail} File}
\smallskip

The \Pg{.detail} file begins with the command that invoked the program and
a list of the parameters
used by the program.  Here is a sample:
\BSV
\begin{verbatim}
Command:  /fs/szgenefinding/Glimmer3/bin/glimmer3 -o 50 -g 110 -t 30 -b iterated.motif -P 
0.603,0.338,0.059 tpall.fna iterated.icm iterated

Sequence file = tpall.fna
Number of sequences = 1
ICM model file = iterated.icm
Excluded regions file = none
List of orfs file = none
Input is NOT separate orfs
Independent (non-coding) scores are used
Circular genome = true
Truncated orfs = false
Minimum gene length = 110 bp
Maximum overlap bases = 50
Threshold score = 30
Use first start codon = false
Start codons = atg,gtg,ttg
Start probs = 0.603,0.338,0.059
Stop codons = taa,tag,tga
GC percentage = 52.8%
Ignore score on orfs longer than 799
\end{verbatim}
\ESV

Following that, for each sequence in the input file the
fasta-header line is echoed and followed by a list of orfs
that were long enough for \Pg{glimmer3} to score.  Here is
a sample of the beginning of such a section:
\BSV
\begin{verbatim}
>gi|15638995|ref|NC_000919.1| Treponema pallidum subsp. pallidum str. Nichols, complete ge
nome
Sequence length = 1138011

           ----- Start -----           --- Length ----  ------------- Scores -------------
 ID  Frame   of Orf  of Gene     Stop   of Orf of Gene      Raw InFrm F1 F2 F3 R1 R2 R3 NC
        +2       17       20      139      120     117    -4.94     0 99  0  -  0  -  -  0
        +2      140      242      361      219     117     0.99     0 87  0  - 12  -  -  0
        -1      435      417      148      285     267     5.48     2 97  -  -  2  -  -  0
        +2      668      668      790      120     120     2.89     0 99  0  -  -  -  -  0
        -3      899      839      717      180     120    -0.86     1 95  -  -  -  -  1  3
        -1      936      933      808      126     123     0.38    13 78  -  - 13  -  -  8
        -3     1124     1109      918      204     189    -1.32     0 99  -  -  -  -  0  0
0001    +1        4        4     1398     1392    1392     6.61    99 99  -  -  -  -  -  0
        -2     1750     1720     1457      291     261    -0.92     8  -  -  -  -  8  - 91
        -2     1957     1945     1751      204     192    -1.47     1  -  - 70  -  1  - 27
        -3     2078     2063     1908      168     153    -1.88     4  -  - 20  -  -  4 75
        -2     2308     2293     2174      132     117    -0.38     5  -  - 85  -  5  -  9
0002    +3     1542     1641     2756     1212    1113     3.20    99  -  - 99  -  -  -  0
        -3     2807     2774     2616      189     156    -2.08     3  0  -  -  -  -  3 96
\end{verbatim}
\ESV
Below is a description of the columns.  All positions are counted from the beginning of
the sequence with the first base being position~$1$.
\bl{\settowidth{\labelwidth}{Last Column}\leftmargin=\labelwidth \addtolength{\leftmargin}{1em}\labelsep=1em}\RaggedRight
\item[\Pg{ID}]
  An identification number for a potential gene.  Only orfs whose in-frame (\Pg{InFrm})
  score is above the threshold score (set by the \Pg{-t} option) or are longer
  than the ignore-score length have an entry
  in this column.

\item[\Pg{Frame}]
  The reading frame of the orf---positive for forward strand, negative for reverse strand.
  It is determined by the position of the leftmost base of the stop codon:
  \bn
  \exdent
    frame $+1$ if the stop begins in position $1,4,7,\ldots$;
  \exdent
    frame $+2$ if the stop begins in position $2,5,8,\ldots$; 
  \exdent
    frame $+3$ if the stop begins in position $3,5,9,\ldots$; 
  \exdent
    frame $-1$ if the stop begins in position $3,5,9,\ldots$ (so the leftmost base
    is position $1,4,7,\ldots$);
  \exdent
    frame $-2$ if the stop begins in position $4,7,10,\ldots$ (left base position
    $2,5,8,\ldots$);
  \exdent
    frame $-3$ if the stop begins in position $5,8,11,\ldots$ (left base position
    $3,6,9\ldots$).
  \en
  Note that if the genome length is not a multiple of $3$, for genes that wrap
  around the end of the sequence the same rules applied
  to the start codon position will not yield the same reading frame.

\item[\Pg{Start}]
  The positions of the first base of the orf and the first base of the start codon of the
  gene.  Note that the gene start may be different for the same orf in the \Pg{.predict} file.

\item[\Pg{Stop}]
  Position of the last base of the stop codon.

\item[\Pg{Length}]
  Number of bases in the orf and in the gene.  It does \underline{\emph{NOT}}
  include the bases of the stop codon.

\item[\Pg{Raw} Score]
  This is 100 times the per-base log-odds ratio of the in-frame coding ICM score to the
  independent (\ie, non-coding) model score.  It gives a rough quantification to how
  well an orf scores that can be compared between any two orfs.
  
\item[\Pg{InFrm} Score]
  The normalized (to the range $ 0\ldots 99$) score of the gene in its
  reading frame.  This is just the appropriate-frame value among the next
  six scores.

\item[Frame Scores]
  The normalized (to the range $ 0\ldots 99$) score of the gene in each reading frame.
  A ``\Pg{-}'' indicates the presence of a stop codon in that reading frame.
  The normalization compares only scores without stop codons and the independent
  (non-coding) \Pg{NC} score.  If the orf is sufficiently long, \ie, longer than
  the value stated in ``\Pg{Ignore score on orfs longer than}\ldots'',
  the score is not used.

\item[\Pg{NC} Score]
  The normalized independent (\ie, non-coding or intergenic) model score.  This model
  is adjusted for the fact that the orf, by definition, has no in-frame stop codons.

\item[\Pg{EDR} Score]
  An additional column of scores is produced if the \Pg{-E}~option
  is specified.
  This is the entropy-distance ratio, \ie, the ratio of the distance
  of the amino-acid distribution from a positive model to the distance
  from a negative model.  Scores below $1.0$ are more likely to be genes;
  scores above $1.0$ less likely to be genes.
  It is not currently used in the scoring process.
\el

\smallskip
\noindent\textbf{\Pg{.predict} File}
\smallskip

This file has the final gene predictions.  It's format is the fasta-header
line of the sequence followed by one line per gene.  Here is a sample of the
beginning of such a file:
\BSV
\begin{verbatim}
>gms:3447|cmr:632 chromosome 1 {Mycobacterium smegmatis MC2}
orf00001      499     1692  +1    13.14
orf00004     1721     2614  +2    14.20
orf00006     2624     3778  +2    10.35
orf00009     3775     4359  +1     9.34
\end{verbatim}
\ESV
The columns are:
\bl{\settowidth{\labelwidth}{Column 1}\leftmargin=\labelwidth \addtolength{\leftmargin}{1em}\labelsep=1em}\RaggedRight
\item[Column 1]
  The identifier of the predicted gene.  The numeric portion matches the
  number in the \Pg{ID} column of the \Pg{.detail} file.

\item[Column 2]
  The start position of the gene.

\item[Column 3]
  The end position of the gene.  This is the last base of the stop codon, \ie,
  it includes the stop codon.

\item[Column 4]
  The reading frame.

\item[Column 5]
  The per-base ``raw'' score of the gene.  This is slightly different from the
  value in the \Pg{.detail} file, because it includes adjustments for the
  PWM and start-codon frequency.
\el

\subsection{\Pg{long-orfs} Program}

This program identifies long, non-overlapping open reading frames (orfs)
in a DNA sequence file.  These orfs are very likely to contain genes,
and can be used as a set of training sequences for the \Pg{build-icm}
program.  More specifically, among all orfs longer than a minimum length
$\ell$, those that do not overlap any others are output.  The start
codon used for each orf is the first possible one.  The program, by
default, automatically determines the value $\ell$ that maximizes the
number of orfs that are output.  With the \Pg{-t} option, the initial
set of candidate orfs also can be filtered using entropy distance, which
generally produces a larger, more accurate training set, particularly
for high-GC-content genomes.  Entropy distance is described in~\cite{med1}.

\subsubsection{\Pg{long-orfs} Parameters \& Options}
The format for invoking \,\Pg{long-orfs}\, is:
\bq
  \Pg{long-orfs}\, [\Desc{options}] \Desc{sequence} \Desc{output}
\eq
where \Desc{sequence} is the name of the file containing the DNA sequence
to be analyzed and \Desc{output} is the name of the output file of
coordinates.  \Desc{sequence} may contain only one sequence.
If \Desc{output} is ``\Pg{-}'', then the output is directed to
standard output.

Possible \Desc{options} are:
\bl{}\RaggedRight
\exdent
  \verb`-A` \Desc{codon-list} \enskip or \enskip \verb`--start_codons` \Desc{codon-list}

  Specify allowable start codons as a comma-separated list.
  Sample format:  \,\verb`-A atg,gtg`\,.
  The default start codons are \Pg{atg}, \Pg{gtg} and \Pg{ttg}.

\exdent
  \verb`-E` \Desc{filename} \enskip or \enskip \verb`--entropy` \Desc{filename}

  Read entropy profiles from \Desc{filename}.  The format is one header
  line, then 20 lines of 3 columns each, which is the format produced
  by the program \Pg{entropy-profile} with the \Pg{-b} option.
  The columns are amino acid,
  positive entropy, and negative entropy, respectively.  Rows must be in
  alphabetical order by amino acid code letter.

  The entropy profiles are used only if the \Pg{-t} option is specified.

\exdent
  \verb`-f` \enskip or \enskip \verb`--fixed`

  Do \underline{\emph{NOT}} automatically calculate the minimum gene
  length that maximizes the number or length of output regions, but
  instead use either the value specified by the \Pg{-g} option or
  else the default, which is 90.

\exdent
  \verb`-g` \Desc{n} \enskip or \enskip \verb`--min_len` \Desc{n}

  Set the minimum gene length to \Desc{n} nucleotides.  This does not include
  the bases in the stop codon.

\exdent
  \verb`-h` \enskip or \enskip \verb`--help`

  Print the usage message.

\exdent
  \verb`-i` \Desc{filename} \enskip or \enskip \verb`--ignore` \Desc{filename}

  \Desc{filename} specifies regions of bases that are off 
  limits, so that no bases within that area will be examined.
  The format for entries in this file is described above for
  the same option in the \Pg{glimmer3} program.

\exdent
  \verb`-l` \enskip or \enskip \verb`--linear`

  Assume a linear rather than circular genome, \ie, there will
  be no ``wraparound'' genes with part at the beginning of the sequence
  and the rest at the end of the sequence.

\exdent
  \verb`-L` \enskip or \enskip \verb`--length_opt`

  Find and use as the minimum gene length the value that maximizes the
  total \underline{\emph{length}} of non-overlapping genes, instead of
  the default behaviour, which is to maximize the total \underline{\emph{number}}
  of non-overlapping genes.

\exdent
  \verb`-n` \enskip or \enskip \verb`--no_header`

  Do not include the program-settings header information in the
  output file.  With this option, the output file will contain
  only the coordinates of the selected orfs.

\exdent
  \verb`-o` \Desc{n} \enskip or \enskip \verb`--max_olap` \Desc{n}

  Set the maximum overlap length to \Desc{n}.  Overlaps of this
  many or fewer bases between genes are not regarded as overlaps.

\exdent
  \verb`-t` \Desc{x} \enskip or \enskip \verb`--cutoff` \Desc{x}

  Only genes with an entropy distance score less than \Desc{x} will
  be considered.  This cutoff is made before any subsequent steps
  in the algorithm.

\exdent
  \verb`-w` \enskip or \enskip \verb`--without_stops`

  Do \underline{\emph{NOT}} include the stop codon in the region
  described by the output coordinates.  By default it is included.

\exdent
  \verb`-z` \Desc{n} \enskip or \enskip \verb`--trans_table` \Desc{n}

  Use Genbank translation table number \Desc{n} to specify stop codons.

\exdent
  \verb`-Z` \Desc{codon-list} \enskip or \enskip \verb`--stop_codons` \Desc{codon-list}

  Specify allowable stop codons as a comma-separated list.
  Sample format:  \verb`-Z tag,tga`.
  The default stop codons are \Pg{tag}, \Pg{tga} and \Pg{taa}.
\el

%%%%%%%%%%%%%%%%%%%%%%%%%%%%%%%%%%%%%%%%%%%%%%%%%%%%%%%%%%%%
\section{Versions}

\subsection{Version~0.10}
  \bi\RaggedRight
  \item    
    Initial release.
  \ei

\raggedright
\bibliographystyle{alpha}
\bibliography{notes}

\end{document}
